\documentclass[./main.tex]{subfiles}

\begin{document}
\chapter*{Conclusion}
\markboth{CONCLUSION}{}
\addcontentsline{toc}{chapter}{Conclusion}

Neutrino physics is a very promising section of particle physics. The discovery of neutrino oscillations was an important milestone as a non-zero neutrino mass was proved. To explain the nature of the neutrino mass, physics beyond the Standard Model is essential. This thesis is concentrated on the search for the new physics and the upgrade of the near detector of the long-baseline accelerator neutrino experiment T2K (Tokai-to-Kamioka) hosted in Japan.

Modern accelerator neutrino experiments are performing precise measurements of the neutrino oscillation parameters. The most interesting problem is the search for CP--violation in the neutrino oscillations as this phenomena may be a hint for the explanation of the baryon asymmetry of the Universe. The main limitation of the current experiments is the systematic uncertainties of the flux predictions and neutrino interaction models.
\setcounter{section}{0}
\section{T2K near detector upgrade}
The near detector of the T2K experiment ND280 provides a remarkable reduction of uncertainties from 14\% down to 7\% with flux and neutrino cross--section constraints. But its performance may be further improved. The current setup has a limited acceptance for high-angle tracks. The energy threshold of hadron detection is high, thus not all the secondary particles from neutrino interactions are detected and reconstructed neutrino energy is smeared. To gain accuracy of the neutrino interaction measurements an upgrade of the ND280 is ongoing. A combination of the fine--grained scintillator detector (FGD) and vertical Time Projection Chambers (TPC) has been shown to provide good performance in particle tracking, charge and particle identification, momentum measurements.

A new fully active fine--grained scintillator detector SuperFGD will be installed as an additional neutrino target. It is made from 1 cm${}^3$ cubes and performs tracking in 3D. The dimensions of the new detector are 192$\times$184$\times$56 cm${}^3$ and a fiducial mass is ~2 tons. Fine granularity reduces the threshold for hadron and muon detection. Nearly all the charged secondary particles from neutrino interactions are going to be detected. Thus the accuracy of neutrino interaction measurements is increased.

The production and assembly of the new detector are complicated. A method for the SuperFGD assembly has been developed and tested. It was demonstrated that the whole detector may be assembled in this way. The loose structure of cubes will provide their self-alignment during the fishing line insertion in three dimensions. Fragile fibers are inserted only at the last step of the detector contraction.

A simulation toolkit for the SuperFGD detector was developed. Detector response simulation was calibrated with the data taken during the beamtest. The light yield for different event topologies was estimated. The simulation of the optical cross-talk between the cubes was done. With such a simulation it was demonstrated that the detector can separate different particles with the measurements of the deposited energy. The dynamic range of the MPPCs is large enough (2668 pixels) to measure the stopping protons that will provide at most 600 photoelectrons.

New detectors will provide more detailed information about neutrino interactions, thus new analysis methods have to be developed. A new method of anti-neutrino energy reconstruction using the neutron time of flight energy measurements in SuperFGD was proposed. With the simulation toolkit of the SuperFGD, it was demonstrated that with this additional information, the interaction over Hydrogen can be separated from the interactions over Carbon. Hence we can select a sample of neutrino interactions free from nuclear effects and reconstruct neutrino energy in a more precise way. This method will allow us to constrain both flux normalization and shape that is critical for precise measurements of the $\delta_{CP}$ phase.

Two new TPCs will be installed above and below the SuperFGD. They will provide precise tracking, identification of the charge and particle type, accurate momentum measurements. Resistive Micromegas will be used as a sensitive detector. A resistive foil over the usual Micromegas detector is responsible for the charge sharing between sensitive pads and improving the spatial resolution. Therefore the momentum resolution becomes better.

The prototypes of the new detectors were constructed and tested with beams of charged particles. The performance of the TPC detector was found to exceed the expectations. Spatial resolution for tracks that are parallel to pad borders was measured at the level of 200 $\mu$m. That will make possible very precise measurements of the momentum, thus accurate probe of neutrino interaction models. The energy resolution was estimated at the same level as we observed in the existing TPC (10\% for one module). With the new detectors we are going to decrease the systematic uncertainty in the oscillation analysis from 7\% to 4\%.



\section{Search for Heavy Neutral Lepton}
Accelerator neutrino experiments are capable not only to measure the oscillation parameters, but to perform a search of the physics beyond the Standard Model (SM). A hypothesis proposing Heavy Neutral Leptons (HNL) is an extension of the SM that can explain the nature of the neutrino mass, existence of the Dark Matter and baryon asymmetry of the Universe. A search of the HNL decays was performed with the near detector of the T2K experiment. Heavy neutrinos are expected to be produced in the meson decay. With T2K beamline, intense beam of kaons is produced. Thus the search of the HNL with masses $M_N < 500$ MeV/c${}^2$ is possible. We concentrated on the search of the two body decays $N\to\mu\pi$, $N\to e\pi$ and dimuon mode $N\to\mu\mu\nu$. The fiducial volume of the three TPCs was used in this analysis. Few neutrino interactions are expected in the gas with atmospheric pressure, comparing to the scintillator detectors. Thus the background is naturally suppressed. The ND280 tracker already demonstrated good performance and ability to identify particle's charge and type. That is very helpful in the searches for neutral particle decay.

Expected signal events were simulated in the detector. Kinematic spectra of the daughter particles were studied in order to separate them from the neutrino interactions in gas. A cut sequence was developed to suppress the background from the neutrino interactions. The final efficiency for the signal detection was observed at the level of 20\% and nearly no background was expected. The systematic uncertainties from the flux predictions and detector efficiency were estimated.

After data unblinding only one event in the $N\to\mu\mu\nu$ mode was observed. A strong upper limit on the existence of the heavy neutrino with masses $M_N < 500$ MeV/c${}^2$ was set. The best result was achieved at the high mass. Mixing elements were constrained with $\left|U_e\right|^2 < 2\times10^{-9}$ for $M_N > 420$ MeV/c${}^2$ and $\left|U_e\right|^2 < 3\times10^{-9}$ for $M_N > 350$ MeV/c${}^2$. This result improved the previous analysis done by the PS191 experiment.

\begin{comment}
\section{Abstract}
\subsection{English}
Accelerator neutrino experiments are now focusing on the precision measurements of the neutrino oscillation parameters and search for the CP--violation in the lepton sector. The systematic uncertainty of the flux and neutrino interaction models are the main limitations of the sensitivity of the experiments. The T2K experiment is upgrading its near detector complex (ND280) to reduce these systematics. Acceptance of the setup is going to be enlarged to 4$\pi$ angle and the particle detector thresholds are going to be reduced. After the upgrade, the systematic uncertainties in the oscillation analysis of the T2K experiment will be reduced from 7\% down to 4\%.

New detectors are going to be installed in ND280. A new 3D fine-grained scintillator detector is a target for neutrino interactions. The low threshold for hadron and muon detection will provide a precise probe of neutrino interaction models. Time Projections Chambers are put above and below the target and serve for acceptance of the high--angle tracks, particle identification, momentum measurements. The resistive Micromegas technology will be used to improve the spatial resolution of the TPC detectors.

Prototypes of the new detectors were built and tested with beams of charged particles. A good performance that meets the requirements for the precise physics measurements was observed. Full--size detectors are under construction and integration in the ND280 is planned in 2022.

New methods of physics analysis are under development. It was proved that with new setup more precise measurements of neutrino interactions are possible.



The discovery of the neutrino oscillations indicates a non-zero mass of the neutrino. The explanation of such a phenomenon requires new physics beyond the Standard Model. The existence of Heavy Neutral Leptons (HNL) is a promising hypothesis that provides a minimal natural explanation of the neutrino mass, the existence of Dark Matter, and baryon asymmetry of the Universe.

Accelerator neutrino experiments can perform a search for the GeV scale heavy neutrinos. A search of the HNL decays was performed with the near detector of the T2K experiment. The active volume of the gaseous TPCs was used as the expected background from the neutrino interactions is expected to be small compared to scintillator detectors. The high tracking performance, charge, and particle identification make the ND280 detector very sensitive to HNL decay search.

A simulation of the signal sample was performed. A selection of the HNL decays and reduction of the background from neutrino interactions was studied, systematic uncertainties of the analysis were estimated. A search for HNL decays was performed with T2K data set (2010-2017). No significant signs of the exotic particles were observed and a strong upper limit on their existence was set.

\subsection{Francais}

Les expériences sur les neutrinos accélérateurs se concentrent maintenant sur les mesures de précision des paramètres d'oscillation des neutrinos et recherchent la violation de CP dans le secteur leptonique. L'incertitude systématique des modèles de flux et d'interaction neutrino est la principale limitation de la sensibilité des expériences. L'expérience T2K modernise son complexe de détecteurs proches (ND280) afin de réduire cette systématique. L'acceptation de la configuration va être élargie à un angle de 4$\pi$ et les seuils du détecteur de particules vont être réduits. Après la mise à niveau, les incertitudes systématiques dans l'analyse des oscillations de l'expérience T2K seront réduites de 7 \% à 4 \%.

De nouveaux détecteurs vont être installés dans le ND280. Un nouveau détecteur scintillateur 3D à grains fins est une cible pour les interactions avec les neutrinos. Un seuil bas pour la détection des hadrons et des muons fournira une sonde précise des modèles d'interaction des neutrinos. Les champers à projections temporelles sont placées au-dessus et au-dessous de la cible et servent à détecteur des particules chargés à angle élevé, à l'identification des particules et aux mesures de moment. La technologie résistive Micromegas sera utilisée pour améliorer la résolution spatiale des détecteurs TPC.

Des prototypes des nouveaux détecteurs ont été construits et testés au faisceau de particules chargées. Une bonne performance répondant aux exigences de mesures physiques précises a été observée. Des détecteurs de grande taille sont en construction et leur intégration dans le ND280 est prévue en 2022.

De nouvelles méthodes d'analyse physique sont en cours de développement. Il a été prouvé qu'avec une nouvelle configuration, des mesures plus précises des interactions des neutrinos sont possibles.



La découverte des oscillations des neutrinos indique une masse non nulle du neutrino. L'explication d'un tel phénomène nécessite une nouvelle physique au-delà du modèle standard. L'existence de leptons neutres lourds (HNL) est une hypothèse qui fournit une explication naturelle minimale de la masse de neutrinos, l'existence de la matière noire et l'asymétrie baryonique de l'Univers.

Les expériences avec les neutrinos accélérateurs peuvent effectuer une recherche des neutrinos lourds à l'échelle du GeV. Une recherche des désintégrations HNL a été effectuée avec le détecteur proche de l'expérience T2K. Le volume actif des TPC gazeux a été utilisé car le bruit de fond attendu des interactions neutrinos est faible par rapport aux détecteurs à scintillateur. Le détecteur ND280 est très sensible à la recherche de désintégration HNL.

Une simulation de l'échantillon de signal a été réalisée. Une sélection des désintégrations HNL et la réduction du signal du fond des interactions avec les neutrinos ont été étudiées, les incertitudes systématiques de l'analyse ont été estimées. Une recherche des désintégrations HNL a été effectuée avec l'ensemble de données T2K (2010-2017). Aucun signe fiable de particules exotiques n'a été observé et une limite supérieure forte a été placée.
\end{comment}
\end{document}